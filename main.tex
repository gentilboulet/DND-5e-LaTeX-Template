%%%%%%%%%%%%%%%%%%%%%%%%
% DND 5e LaTeX Template example
% Created by GentilBoulet
% Nov 22015
%%%%%%%%%%%%%%%%%%%%%%%%
\documentclass[10pt]{article}

\usepackage{lipsum}         % Filler text
\usepackage{multicol}       % Two cols
\usepackage[cm]{fullpage}   % Small margins
\setlength{\columnsep}{1cm}
\usepackage{bookman}        % Closest built-in font I could find
\usepackage{dnd5e}

% Start document
\begin{document}
\begin{multicols}{2}
\fontfamily{ppl}\selectfont % Set text font
% Your content goes here
\section*{Test Section}
\lipsum[1]
\subsection*{Test Subsection}
    \begin{commentbox}{Neat Green Box!}
        \lipsum[1]
    \end{commentbox}
    \lipsum[3]
    \noindent
    
    \begin{lmss} % Switch font
    \rowcolors{1}{bgtan}{commentgreen} % Alternate colors
    \begin{tabularx}{\linewidth}{XX}
        \textbf{Table head}  & \textbf{Table head} \\
        Some value  & Some value \\
        Some value  & Some value \\
        Some value  & Some value
    \end{tabularx}
    \end{lmss}

    \begin{monster}{Monster Foo}{Small metasyntatic variable (golbinoid), neutral evil}
      \basics[%
        armorclass = 12,
        hitpoints  = 16 (3d8 + 3),
        speed      = 50 ft
      ]
      \fancyhrule      
      \stats[%
        str = 3,
        dex = 6,
        con = 9,
        int = 12,
        wis = 15,
        cha = 18,
        ]
      \fancyhrule
      \details[%
        % If you want to use commas in these sections, enclose the
        % description in braces.
        % I'm so sorry.
        languages = {Common Lisp, Erlang},
        challenge = 1/8 (25 XP),
      ]
      \fancyhrule
      \monstersection{Actions}
      \monsteraction{KungFoo}{The foo monster makes two attackes with its kungfu}
      \monsteraction{KungFoo2}{The foo monster makes two attackes with its kungfu}
      \monsteraction{KungFoo3}{The foo monster makes two attackes with its kungfu}
      \monsteraction{Unarmed Strike}{\textit{Melee Weapon Attack:} +4 to hit, reach 5ft., one target.\textit{Hit:} 6(1d8+2) blundeoning damage plus 9 (2d8) psychic damage. This is a magic weapon attack.}
      \monsteraction{Spellcasting.}{The archmage is an 18th-level spellcaster. Its spellcasting ability is Intelligence (spell save DC 17, +9 to hit
      	with spell attacks). The arch mage can cast disguise self and
      	invisibility at will and has the following wizard spells prepared:\\
      	\vspace{1ex}
      	Cantrips (at will):.fire bolt, light, mage hand, prestidigitation,
      	shocking grasp\\
      	1st level (4 slots): detect magic, identify, mage armor*,
      	magic missile\\
      	2nd level (3 slots): detect thoughts, mirror image, misty step\\
      	3rd level (3 slots): counterspeii,Jly, lightning bolt\\
      	4th level (3 slots): banishment, fire shield, stoneskin*\\
      	5th level (3 slots): cone of cold, scrying, wall of force\\
      	6th level .(1 slot): globe of invulnerability\\
      	7th level (l slot): teleport\\
      	8th level (l slot): mind blank*\\
      	9th level (l slot): time stop\\
      	\vspace{1ex}
      	{}* The archmage casts these spells on itself before combat.}
      
     \monstersection{Legendary}
     The foo monster can take 3 legendary actions, choosing from the options below. Only one legendary action option can be used at a time and only at the end of another creature's turn. The foo monster regains spent legendary actions at the start of its turn.\\ \vspace{1ex}
     \monsteraction{Move}{The foo monster moves up its speed without provoking opportunities attacks}
     \monsteraction{Unarmed Strike}{The foo monster makes one unarmed strike}
    \end{monster}
    \lipsum
% End document
\end{multicols}
\end{document}
